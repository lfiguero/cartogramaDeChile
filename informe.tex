\documentclass[20pt]{report}
\usepackage{amsmath, amssymb,amsmath}

\usepackage[spanish, es-lcroman]{babel}  
\usepackage[spanish]{babel}
\usepackage{multicol}
\usepackage{graphicx}
\usepackage{makeidx}
\usepackage{multicol}
 \cleardoublepage
\usepackage{graphicx}
\usepackage{float}
\providecommand{\abs}[1]{\lvert#1\rvert}
\providecommand{\norm}[1]{\lVert#1\rVert}
\usepackage{listings}
\usepackage{fancybox}
\usepackage{cancel}
%\usepackage{float}
\usepackage{wrapfig}
\parindent=2.1pc
\setlength{\oddsidemargin}{0.5cm} \setlength{\evensidemargin}{0cm}
\setlength{\textwidth}{16cm} \setlength{\textheight}{23cm}
\setlength{\topmargin}{-.5cm}
\begin{document}

\begin{figure}[H]
  \scalebox{0.5}{
    \includegraphics[scale=0.2]{1.jpg} 
    %\caption{Descripci\'on si se desea} 
    }
\end{figure}

\begin{figure}[H]
  \scalebox{0.5}{
  
    \includegraphics[scale=0.5]{udec4.jpg} 
    %\caption{Descripci\'on si se desea} 
    }
    \centering
\end{figure}


\topmargin=-1.6cm

%%%%%%%%%%%%%%%%%%%%% ENCABEZADO %%%%%%%%%%%%%%%%%%%%%%%%%%%%%%%%

\vspace{1cm}

\noindent\rule{15cm}{.5pt}

\begin{center}
\noindent{\huge\textbf{{UNIVERSIDAD DE CONCEPCI\'ON}}}\\
\noindent{\Large\textbf{{FACULTAD DE CIENCIAS F\'ISICAS Y MATEM\'ATICAS }}}\\

{DEPARTAMENTO DE INGENIER\'IA EN MATEM\'ATICA}
\end{center}
\begin{center}
\noindent\rule{15cm}{.5pt}\\
\vspace{2cm}
\textbf{\Large{ CARTOGRAMA	 DE CHILE}}\\

\Large{Sebasti\'an Moraga Scheuermann}\\

\Large{Profesor Leonardo Figueroa}\\

\end{center}
%\vspace{0.5cm} \centerline{\bf \large } 



 %\centerline{\bf \large Ingenier\'ia Civil Matem\'atica}\\
 %\centerline{\bf \large Universidad de Concepci\'on}
%%%%%%%%%%%%%%%%%%%%%%%%%%%%%%%%%%%%%%%%%%%%%%%%%%%%%%%%%%%%%%%%%
 \vspace{.2cm}
  \noindent
%%%%%%%%%%%%%%%%%%%%%%%%%%%%%%%%%%%%%%%%%%%%%%%%%%%%%%%%%%%%%%%%%
\begin{itemize}
\item [\bf ]{\bf    }
\topmargin=-1.6cm


\end{itemize}
\topmargin=-1.6cm
\vspace{3cm}
\noindent 



%\rule{16cm}{.5pt} 



\pagebreak


 \tableofcontents % indice de contenidos


\chapter{Introducci\'on}\label{cap.introduccion}

\pagenumbering{arabic} % para empezar la numeraci\'on con n\'umeros
Un cartograma es un mapa que muestra gr\'aficamente datos de un lugar geogr\'afico alterando su  imagen,  por medio de alg\'un patr\'on o medida de \'area que sea identificable  y f\'acilmente interpretable.\\
 El siguiente texto trata sobre  la creaci\'on de un cartograma de Chile basado en el m\'etodo de difusi\'on de Gastner/Newman [2004], que adapta mapas seg\'un la densidad de una regi\'on sin alterar sus relaciones  con otras regiones.
 \\
 \\
A medida que ha pasado el tiempo se han creado varios m\'etodos de c\'omo crear un cartograma,  pero todos de ellos con alg\'un problema como fuerte dependencia de los ejes coordenados que se usan, o simplemente problemas al tener zonas que se sobreponen entre otros.
\\
\\
    
Un archivo .shp  es un archivo con informaci\'on sobre alguna regi\'on,  l\'imites, adem\'as de contener las delimitaciones de las zonas puede almacenar  distintas variables como poblaci\'on, densidad,  o cualquier otro atributo que se le quiera asignar a alguna zona. Este es el archivo con el cual  trabajaremos para crear el cartograma, adem\'as fue usado el programa QGIS 2.18.1 para agregar atributos a un archivo division\_comunal.shp y as\'i poder trabajar con \'el. 
\pagebreak



\chapter{Enunciado }\label{cap.introduccion}
\section{ Inter\'es}
\begin{itemize}
\item Sea $\Omega$ la superficie de la Rep\'ublica de Chile y sea $\rho$ una aproximaci\'on de la funci\'on densidad-humana definida sobre $\Omega$. Queremos tener un cartograma de Chile a partir de la informaci\'on regida del censo 2012
\\
\begin{figure}[H]
\begin{center}
\includegraphics[width=10cm, height=10cm]{Chile.png}
\vspace{-0.5cm} %Espacio vertical negativo para pegar mas el caption de la figura a la propia figura
\caption{Chile continental}
\label{Label para referencia}
\end{center}
\end{figure}



\pagebreak
\label{cap.introduccion}\section{Base te\'orica}

Vemos este proyecto es una aplicaci\'on del trabajo realizado por en "Diffusion-based method for producing
density-equalizing maps" creado por   "Michael T. Gastner and M. E. J. Newman" entonces es claro que su m\'etodo ser\'a explicado en lo que sigue y se mostrar\'an los resultados obtenidos. \\
B\'asicamente lo que ellos realizan es describir la poblaci\'on  por una funci\'on densidad $\rho (r)$ donde $r$ es la posici\'on geogr\'afica, y de ello vamos a obtener la difusi\'on. Tal como se describe en el paper, a medida que $t\longrightarrow \infty$  la densidad completa del mapa geogr\'afico se vuelve  uniforme y el desplazamiento total determina la proyecci\'on necesaria para obtener el cartograma.
El procedimiento completo est\'a explicado en su paper, sin embargo resaltaremos lo m\'as importante,  que es la construcci\'on de la soluci\'on y  los m\'etodos usados.
\\
En difusi\'on estandar la densidad actual est\'a dada por
\begin{equation}
J= v(r,t)\rho (r,t)
\end{equation}
Donde $v(r,t)$ y $\rho(r,t)$ son la velocidad y la densidad respectivamente, con $r$ y $t$ la posici\'on y el tiempo. Adem\'as el gradiente del campo densidad es

\begin{equation}
J=-\nabla \rho
\end{equation}
lo que indica que el flujo  va en direcci\'on  de  m\'as densidad a donde hay menos, en el paper se indica que podemos omitir  una constante de difusi\'on  que podr\'ia aparecer en la ecuaci\'on anterior. Entonces la difusi\'on de poblaci\'on es conservada localmente  as\'i que  se tiene 
\begin{equation}
\nabla \cdot J + \dfrac{\partial \rho}{\partial t} =0
\end{equation}
Combinando estas ecuaciones se tiene la ecucaci\'on de difusi\'on:
\begin{equation}
\nabla ^2 \rho - \dfrac{\partial \rho }{\partial t}=0
\end{equation}
y la expresi\'onde del  campo velocidad en t\'erminos de la densidad de poblaci\'on es:
\begin{equation}
v(r,t)= - \dfrac{\nabla \rho}{\rho}
\end{equation}


Para calcular el cartograma  se debe resolver la ecuaci\'on (2.4) para $\rho(r,t)$. Bajo las condiciones  mencionadas en el paper se tiene que se resolvi\'o la ecuaci\'on de difusi\'on en  espacios de Fourier,  el cal es diagonal y se vuelve a la transofrmada antes de integrar sobre el campo de velocidades, con esto y  con las condiciones de borde del tipo  Newmann, la base de Fourier es de cosenos  en cuyo caso la soluci\'on a la ecuaci\'on de difusi\'on tienen la forma
\begin{equation}
\rho (r,t)= \dfrac{4}{L_x L_y} \sum_k {\overline{\rho}} (k) \cos(k_x x) \cos(k_y y)exp(-k^2 t)
\end{equation}
Donde $L_x,L_y$ son las dimensiones de la zona rectangular  donde se encontrar\'a el mapa, en el cual son  lo suficientemente m\'as amplias que las propias coordenadas de las regiones para efectos de la difusi\'on. \\
La suma est\'a calculada sobre los vectores $k=(k_x,k_y)=2 \pi (m/L_x, n/L_y)$ con m, n enteros no negativos y $\overline{\rho}(k$ es la transformacion discreta del coseno de $\rho (r, t=0)$:
\begin{equation}
\overline{\rho}(k)=\dfrac{1}{4}(\delta_{k_x,0}+1)(\delta_{k_y,0}+1)\times \int_0^{L_x} \int_0^{L_y} \rho (r,0)\cos(k_x x)\cos(k_y y) dx dy
\end{equation}


\pagebreak
Donde $\delta_{i,j}$ es el delta de Kronecker. El campo de velocidades $v$ es facilmente calculable de la ecuaci\'on (2.5) y (2.6)  y tiene componentes
\begin{equation}
v_x(r,t)=\dfrac{\sum_k {k_x\overline{\rho}} (k) \sin(k_x x) \cos(k_y y)exp(-k^2 t)}{\sum_k {\overline{\rho}} (k) \cos(k_x x) \cos(k_y y)exp(-k^2 t)}
\end{equation}
\begin{equation}
v_y(r,t)=\dfrac{\sum_k {k_y\overline{\rho}} (k) \cos(k_x x) \sin(k_y y)exp(-k^2 t)}{\sum_k {\overline{\rho}} (k) \cos(k_x x) \cos(k_y y)exp(-k^2 t)}
\end{equation}
Ecuaciones (2.7), (2.8), (2.9) pueden ser r\'apidamente evaluadas usando la r\'apida transformada de Fourier (FFT) y su inversa respectivamente,  cada una  en un tiempo del orden $L_x L_y log(L_x L_y)$. Entonces se usa  el campo de velocidades restante para integrar la ecuaci\'on 
\[r(t)= r(0) +\int_0^t v(r,\tau)d\tau\]
Que es el desplazamiento acumulativo de cualquier punto del mapa al tiempo $t$.
\\
Esto genera una ecuaci\'on  no Volterra no lineal de segundo tipo que puede ser resuelta num\'ericamente por m\'etodos est\'andar.



Esta es la base te\'orica en la que se bas\'o el proyecto del cartograma para Chile, as\'i pues se encontr\'o el software ScapeToad  creado por Dominique Andrieu, Christian Kaiser y Andr\'e Ourednik que  hace uso de exactamente  del dise\~no para crear cartogramas de Michael T. Gastner and M. E. J. Newman, el cual se alimenta de un archivo .shp  que contiene los datos, la forma  y de limitaciones de regiones que  queremos cartografiar, adem\'as de la densidad de cada regi\'on. Es por ello que ocupando el programa "QGIS Las" se retiraron zonas de chile que no eran pr\'oximas a Chile continental, tales como Isla de Pascua o Juan Fern\'andes pues al estar alejadas no producen distorsi\'on en lo que es  Chile continental. \\
Luego el archivo original division\_comunal.shp tampoco contaba con la densidad de cada regi\'on, por lo que se debi\'o  manualmente  crear ese campo de valores y establecerlos de acuerdo al censo sacado de Wikipedia. \\
As\'i  se fue completando este archivo arduamente hasta obtener todo lo necesario, luego simplemente se dej\'o correr el programa ScapeToad creando una matriz de $2000$ filas y $333$ columnas sobre la imagen de Chile, al cual a cada nodo con  coordenadas $(x,y)$ se le asigna un valor de densidad, el cual   hace uso de la t\'ecnica antes descrita. El programa funcion\'o por 12.5 horas continuas.
\pagebreak

\label{cap.introduccion}\section{Resultados num\'ericos} 
En la secci\'on siguiente tenemos los resultados de los cartogramas de Chile
\begin{figure}[H]
\begin{center}
\includegraphics[width=5cm, height=5cm]{Chile.png}
\vspace{-0.5cm} %Espacio vertical negativo para pegar mas el caption de la figura a la propia figura
\caption{Chile continental sin difusi\'on}
\label{Label para referencia}
\end{center}
\end{figure}
\begin{figure}[H]
\begin{center}
\includegraphics[width=5cm, height=5cm]{1.png}
\vspace{-0.5cm} %Espacio vertical negativo para pegar mas el caption de la figura a la propia figura
\caption{Chile continental con difusi\'on}
\label{Label para referencia}
\end{center}
\end{figure}
\begin{figure}[H]
\begin{center}
\includegraphics[width=5cm, height=5cm]{malla.png}
\vspace{-0.5cm} %Espacio vertical negativo para pegar mas el caption de la figura a la propia figura
\caption{grilla sobre chile}
\label{Label para referencia}
\end{center}
\end{figure}
\begin{figure}[H]
\begin{center}
\includegraphics[width=5cm, height=5cm]{center.png}
\vspace{-0.5cm} %Espacio vertical negativo para pegar mas el caption de la figura a la propia figura
\caption{Centro de Chile: Santiago}
\label{Label para referencia}
\end{center}
\end{figure}
\begin{figure}[H]
\begin{center}
\includegraphics[width=5cm, height=5cm]{santiago.png}
\vspace{-0.5cm} %Espacio vertical negativo para pegar mas el caption de la figura a la propia figura
\caption{Centro de Chile: Comuna de Santiago}
\label{Label para referencia}
\end{center}
\end{figure}

\begin{figure}[H]
\begin{center}
\includegraphics[width=5cm, height=5cm]{talca.png}
\vspace{-0.5cm} %Espacio vertical negativo para pegar mas el caption de la figura a la propia figura
\caption{Centro de Chile: Talca}
\label{Label para referencia}
\end{center}
\end{figure}
\begin{figure}[H]
\begin{center}
\includegraphics[width=5cm, height=5cm]{nort.png}
\vspace{-0.5cm} %Espacio vertical negativo para pegar mas el caption de la figura a la propia figura
\caption{Norte de Chile}
\label{Label para referencia}
\end{center}
\end{figure}

Finalmente se adjunta un archivo llamado FINALGRID.shp que contiene la grilla sobre Chile por si  el lector desea revisarla, y adem\'as FINALGRID2.shp contiene el cartograma de Chile para ser revisado en detalle. 
\pagebreak

\section{Bibliografia}
\item http:\textbackslash \textbackslash  www.bcn.cl \textbackslash siit\textbackslash mapas\_ vectoriales\textbackslash index\_ html
\item https://es.wikipedia.org/wiki/Anexo:Comunas\_de\_Chile
\item Gastner, M.T. and Newman, M. (2004). Diffusion-based method for producing density equalizing maps. In Proceedings of the National Academy of Sciences of the United States of America, 101(20): 7499-7504.
\item http://scapetoad.choros.ch/index.php
\item Qgis Software
\end{itemize}
\end{document}

